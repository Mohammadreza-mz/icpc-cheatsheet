\textbf{Finding an arbitrary flow:} We make the following changes in the network. We add a new source $s'$
and a new sink $t'$, a new edge from the source $s'$ to every other
vertex, a new edge for every vertex to the sink $t'$, and one edge
from $t$ to $s$. Additionally we define the new capacity function
$c'$ as: \\
• $c'((s', v)) = \sum_{u \in V} d((u, v))$ for each edge $(s', v)$. \\
• $c'((v, t')) = \sum_{w \in V} d((v, w))$ for each edge $(v, t')$. \\
• $c'((u, v)) = c((u, v)) - d((u, v))$ for each edge $(u, v)$ in the
  old network. \\
• $c'((t, s)) = \infty$ \\

If the new network has a saturating flow (a flow where each edge
outgoing from $s'$ is completely filled, which is equivalent to every
edge incoming to $t'$ is completely filled), then the network with
demands has a valid flow, and the actual flow can be easily
reconstructed from the new network. Otherwise there doesn't exist a flow
that satisfies all conditions. Since a saturating flow has to be a
maximum flow, it can be found by any maximum flow algorithm. \\

\textbf{Minimal flow:}
Note that along the edge $(t, s)$ (from the old sink to the old
source) with the capacity $\infty$ flows the entire flow of the
corresponding old network. I.e. the capacity of this edge effects the
flow value of the old network. By giving this edge a sufficient large
capacity (i.e. $\infty$), the flow of the old network is unlimited. By
limiting this edge by smaller capacities, the flow value will decrease.
However if we limit this edge by a too small value, than the network
will not have a saturated solution, e.g. the corresponding solution for
the original network will not satisfy the demand of the edges. Obviously
here can use a binary search to find the lowest value with which all
constraints are still satisfied. This gives the minimal flow of the
original network.